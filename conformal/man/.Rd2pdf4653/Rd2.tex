\documentclass[a4paper]{book}
\usepackage[times,inconsolata,hyper]{Rd}
\usepackage{makeidx}
\usepackage[latin1]{inputenc} % @SET ENCODING@
% \usepackage{graphicx} % @USE GRAPHICX@
\makeindex{}
\begin{document}
\chapter*{}
\begin{center}
{\textbf{\huge Conformal: an R package to calculate prediction errors in the conformal prediction framework}}
\par\bigskip{\large \today}
\end{center}
\inputencoding{utf8}
\HeaderA{conformal}{conformal: an R package to calculate prediction errors in the conformal prediction framework}{conformal}
\aliasA{conformal-package}{conformal}{conformal.Rdash.package}
%
\begin{Description}\relax
conformal permits the calculation of prediction errors in the conformal prediction framework:
(i) p.values for classification, and
(ii) confidence intervals for regression.
\end{Description}
%
\begin{References}\relax
Isidro Cortes <isidrolauscher@gmail.com>.
conformal: an R package to calculate prediction errors in the conformal prediction framework

Norinder et al. J. Chem. Inf. Model., 2014, 54 (6), pp 1596-1603        
DOI: 10.1021/ci5001168       
\url{http://pubs.acs.org/doi/abs/10.1021/ci5001168}

\end{References}
\inputencoding{utf8}
\HeaderA{ConformalClassification}{Conformal Prediction For Classification}{ConformalClassification}
%
\begin{Description}\relax
R class to p.values for individual 
predictions according to the conformal prediction framework.
\end{Description}
%
\begin{Usage}
\begin{verbatim}
ConformalClassification(...)
\end{verbatim}
\end{Usage}
%
\begin{Author}\relax
Isidro Cortes-Ciriano <isidrolauscher@gmail.com>
\end{Author}
%
\begin{References}\relax
Norinder et al. J. Chem. Inf. Model., 2014, 54 (6), pp 1596-1603
DOI: 10.1021/ci5001168
\url{http://pubs.acs.org/doi/abs/10.1021/ci5001168}
\end{References}
%
\begin{SeeAlso}\relax
\code{\LinkA{ConformalRegression}{ConformalRegression}}
\end{SeeAlso}
%
\begin{Examples}
\begin{ExampleCode}
# Optional for parallel training
#library(doMC)
#registerDoMC(cores=4)

data(LogS)

# convert data to categorical
LogSTrain[LogSTrain > -4] <- 1
LogSTrain[LogSTrain <= -4] <- 2
LogSTest[LogSTest > -4] <- 1
LogSTest[LogSTest <= -4] <- 2

LogSTrain <- factor(LogSTrain)
LogSTest <- factor(LogSTest)

algorithm <- "rf"

trControl <- trainControl(method = "cv",  number=5,savePredictions=TRUE, 
                          predict.all=TRUE,keep.forest=TRUE,norm.votes=TRUE)
set.seed(3)
model <- train(LogSDescsTrain, LogSTrain, algorithm,type="classification", 
               trControl=trControl)


# Instantiate the class and get the p.values
example <- ConformalClassification$new()
example$CalculateCVAlphas(model=model)
example$CalculatePValues(new.data=LogSDescsTest)
example$p.values$P.values
example$p.values$Significance_p.values

\end{ExampleCode}
\end{Examples}
\inputencoding{utf8}
\HeaderA{ConformalRegression}{Conformal Prediction for Regression}{ConformalRegression}
\keyword{\textbackslash{}textasciitilde{}ConformalRegression}{ConformalRegression}
%
\begin{Description}\relax
R class to create and visualize confidence intervals for individual 
predictions according to 
the conformal prediction framework.
\end{Description}
%
\begin{Details}\relax
The class ConformalRegression contains the following fields:

(i) PointPredictionModel: stores a point prediction model.
(ii) ErrorModel: stores the error model.
(iii) confidence: stores the level of confidence used to calculate the confidence intervals. This value is defined when instantiating a new class. Values are in the 0-1 range. 
Interpretation: for instance, a confidence level of 0.8 (80\%) means that, at most, 20\% of the confidence intervals will not contain the 
true value.
(iv) data.new: stores the descriptors corresponding to an external set.
(v)  alphas: stores the nonconformity scores calculated for the datapoints used to train the point prediction model (PointPredictionModel) with the method CalculateAlphas.
(vi)  errorPredictions: errors in prediction for an external set calculated with the error model (stored in the field ErrorModel) with the method GetConfidenceIntervals.
(vii) pointPredictions: point predictions for an external set calculated with the point prediction model (stored in the field PointPredictionModel) with the method GetConfidenceIntervals.
(viii)  intervals: numeric vector with the errors in prediction for the external set (data.new) calculated in the conformal prediction framework (not with an error model).
These intervals are calculated when calling the method GetConfidenceIntervals.
(ix) plot: stores a correlation plot for the observed against the predicted values, with individual confidence intervals, for the datapoints in an external set. The plot is a ggplot2 object which can be further customized. The plot is generated with the method CorrelationPlot.

The class ConformalRegression contains the following methods:
(i) initialize: this method is called when you create an instance of the class. The confidence level (field confidence) needs to be defined.
(ii) CalculateAlphas: this method calculates the vector of nonconformity scores for the datapoints in the traning set. These scores (or alphas) are stored in the field alphas. This method requires a point prediction model (argument model),
an error model (argument error\_model), and a nonconformity measure (argument ConformityMeasure), such as \code{\LinkA{StandardMeasure}{StandardMeasure}}.
(iii) GetConfidenceIntervals: this methods calculates confidence intervals for individual predictions in the conformal prediction framework. The methods requires an external set for which the confidence intervals will be calculated.
The dimensionality of these descriptors need to be the same as the one used for the datapoints used to train the point prediction model and the error model. 
The method uses the point prediciton and the error models stored in the fields PointPredictionModel and ErrorModel, respectively.
Confidence intervals are calculated according to \Rhref{{http://pubs.acs.org/doi/abs/10.1021/ci5001168}{Norinder et al. 2014}}{}To define 
Confidence intervals are defined as: point prediction (stored in the field pointPredictions) +/-  the output of the method GetConfidenceIntervals, which is stored in the field intervals.
(iv) CorrelationPlot: this method generates a correlation plot for the observed against the predicted values for an external set, along with individual confidence intervals. 
The plot is stored in the field plot.
\end{Details}
%
\begin{Author}\relax
Isidro Cortes-Ciriano <isidrolauscher@gmail.com>
\end{Author}
%
\begin{References}\relax
Norinder et al. J. Chem. Inf. Model., 2014, 54 (6), pp 1596-1603
DOI: 10.1021/ci5001168
\url{http://pubs.acs.org/doi/abs/10.1021/ci5001168}
\end{References}
%
\begin{SeeAlso}\relax
\code{\LinkA{ConformalClassification}{ConformalClassification}}
\end{SeeAlso}
%
\begin{Examples}
\begin{ExampleCode}
#############################################
### Example
#############################################

# Optional for parallel training
#library(doMC)
#registerDoMC(cores=4)

data(LogS)

algorithm <- "svmRadial"
tune.grid <- expand.grid(.sigma = expGrid(power.from=-10, power.to=-6, power.by=1, base=2), 
                         .C = expGrid(power.from=4, power.to=10, power.by=2, base=2))
trControl <- trainControl(method = "cv",  number=5,savePredictions=TRUE)
set.seed(3)
model <- train(LogSDescsTrain, LogSTrain, algorithm, 
               tuneGrid=tune.grid, 
               trControl=trControl)


# Train an error model
error_model <- ErrorModel(PointPredictionModel=model,x.train=,LogSDescsTrain,
                          savePredictions=TRUE,algorithm=algorithm,
                          trControl=trControl, tune.grid=tune.grid)

# Instantiate the class and get the confidence intervals
example <- ConformalRegression$new()
example$CalculateAlphas(model=model,error_model=error_model,ConformityMeasure=StandardMeasure)
example$GetConfidenceIntervals(new.data=LogSDescsTest)
example$CorrelationPlot(obs=LogSTest)
example$plot



\end{ExampleCode}
\end{Examples}
